%!TEX program = XeLaTeX
\documentclass[UTF8]{ctexart}
\usepackage[a4paper,left=2cm,right=2cm,top=2.5cm,bottom=2.5cm]{geometry}
\usepackage{abstract}
\usepackage{CJK}         % CJK 中文支持
\usepackage{geometry} % 利用 geometry 可以很方便的设置页面的大小。
\usepackage{fancyhdr} % 用 fancyhdr 来设置页眉和页脚十分方便,而且可以在配合 CCT、CJK来设置中文的页眉等。
\usepackage{amsmath,amsfonts,amssymb,graphicx}    % EPS 图片支持
\usepackage{subfig}
\usepackage{indentfirst} % 中文段落首行缩进
\usepackage{mathrsfs} % 用于产生一种数学用的花体字
\usepackage{multirow} 
\usepackage{titletoc}
\usepackage{url}
\usepackage[colorlinks,
            linkcolor=black,
            anchorcolor=blcak,
            citecolor=black]{hyperref}
 
\pagestyle{fancy}
\fancyhead[C]{\zihao{5}  \songti}
\rfoot{\centering \thepage}
\renewcommand{\headrulewidth}{0pt} %改为0pt即可去掉页眉下面的横线
\renewcommand{\footrulewidth}{0pt} %改为0pt即可去掉页脚上面的横线 0.4pt

\newcommand{\makeheadrule}{%
\makebox[0pt][l]{\rule[.7\baselineskip]{\headwidth}{0.8pt}}}
\makeatletter
\renewcommand{\headrule}{%
{\if@fancyplain\let\headrulewidth\plainheadrulewidth\fi
\makeheadrule}}
\makeatother

\title{极客强人挑战赛项目方案}
\author{Super-V $\cdot$ IVSR 视频增强模块组}
\begin{document}
\maketitle
\newpage
\tableofcontents
\newpage

\section{项目简介}
\subsection{项目信息}
\begin{table}[thbp!]
\begin{center}
\begin{tabular}{l|cccc}

  \hline
  % after \\: \hline or \cline{col1-col2} \cline{col3-col4} ...
 \textbf{项目名称} & &&& \textbf{IVSR 视频增强模块组} \\
  \hline
  \textbf{项目实施时间} & &&&2018年9月30日~至~2018年11月1日 \\
  \hline 
  \textbf{项目logo} & &&& 如图所示 \\
  \hline
\end{tabular}
\end{center}
\label{tb:xiangMuXinXi}
\end{table}

\subsection{团队信息}
\begin{itemize}
  \item \textbf{团队名称}:Super-V
\end{itemize}
团队人员详细信息如表\ref{tb:duiYuan}所示:
\begin{table}[thbp!]
\caption{团队人员详细信息}
\begin{center}
\begin{tabular}{l|cccc}
\hline
姓名&部门&职位&邮箱&职能 \\
\hline 
董倩&快视频&产品运营专员&dongqian@360.cn&产品 \\
王聪颖&搜索&机器学习工程师&wangcongying@360.cn&研发 \\
陈政&商业化&算法工程师&chenzheng1@360.cn&研发 \\
刘畅&快视频&服务端开发工程师&liuchang1@360.cn&研发 \\
\hline 
\end{tabular}
\end{center}
\label{tb:duiYuan}
\end{table}

\section{项目目的及意义}
\paragraph{项目背景}
随着社交、短视频信息流、IOT智能家居在现代人们生活中起着越来越重要的作用,在满足信息量的需求之后,人们将需求转向了视频质量。届时视频质量低下成为当今视频流传输的痛点,比如:视频聊天画面不清晰、拍摄的视频模糊不清等、短视频源质量差、监控捕捉画面不佳等,这些原因都会造成用户体验差、流量损失、信息损失、经济损失等。
\paragraph{项目目的}
为了解决以上出现的问题,我们提出一种高效、接口丰富的视频图像质量提升的模块组,其将基于流行的人工智能、深度学习、超分辨率重建等技术进行实现,用于提升线上线下视频的质量,我们的产品最终目的是提高用户体验和留存率,降低信息损失以及经济损失。
\paragraph{项目应用场景}
监控画面、无人机拍摄画面、直播界面、短视频、医学等
\paragraph{项目意义}
提升实时视频聊天的质量,降低由网速造成的干扰;改善由于硬件或者是天气造成的视频模糊;提升视频信息流播放质量;改善监控、行车记录仪、无人机线上线下视频质量;提升直播体验;医学图像的提升等
\section{项目实施基础条件}
本项目中一共有四位同学,其中有一个产品、三个研发工程师,无论从产品以及技术上都有着出色的能力。
\section{项目实施方案}
\subsection{产品形态}
本项目严格上说是提供后端以及前端(非展示)服务。介于模块的测试展示,我们将设计一个实时视频质量提升的工具应用于移动摄像上,不仅如此我们还将提供在线视频图像质量的提升、单帧自然文字识别。
\subsection{技术实现}
为了方便队员的项目实施,本项目已经同步至github中。

项目链接为:\url{https://github.com/Congying-Wang/IVideoSR}
\begin{enumerate}
	\item 测试工具的实现:将基于Object-C以及GPUImage进行开发,实现实时视频质量提升
	\item 线上视频、图像质量提升:基于TensorFlow实现视频、图像提升模型
	\item 单帧文字识别服务:基于Tesseract实现,通过本地数据集以及用户反馈实现动态模型优化
\end{enumerate}
\section{预期成果}
由于队员手头有自己的本职工作,保留完成度为70\%,由于项目意义深远,我们将持续对其进行优化,磨练成一个成熟的产品,那样应用范围将更加广阔,比如生物、刑侦、医学等。
\end{document}